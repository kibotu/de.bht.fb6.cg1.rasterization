%% Based on a TeXnicCenter-Template by Gyorgy SZEIDL.
%%%%%%%%%%%%%%%%%%%%%%%%%%%%%%%%%%%%%%%%%%%%%%%%%%%%%%%%%%%%%

%----------------------------------------------------------
%
\documentclass{report}%
%
%----------------------------------------------------------
% This is a sample document for the standard LaTeX Report Class
% Class options
%       --  Body text point size:
%                        10pt (default), 11pt, 12pt
%       --  Paper size:  letterpaper (8.5x11 inch, default)
%                        a4paper, a5paper, b5paper,
%                       legalpaper, executivepaper
%       --  Orientation (portrait is the default):
%                       landscape
%       --  Printside:  oneside (default), twoside
%       --  Quality:    final (default), draft
%       --  Title page: titlepage, notitlepage
%       --  Columns:    onecolumn (default), twocolumn
%       --  Start chapter on left:
%                       openright(no), openany (default)
%       --  Equation numbering (equation numbers on right is the default)
%                       leqno
%       --  Displayed equations (centered is the default)
%                       fleqn (flush left)
%       --  Open bibliography style (closed bibliography is the default)
%                       openbib
% For instance the command
%          \documentclass[a4paper,12p,leqno]{report}
% ensures that the paper size is a4, fonts are typeset at the size 12p
% and the equation numbers are on the left side.
%
\usepackage{amsfonts}
\usepackage{amssymb}
\usepackage{mathptmx}
\usepackage{times}
\usepackage[latin1]{inputenc}
\usepackage{mathcomp}
\usepackage{amsfonts}
\usepackage{amsmath}
\usepackage{cancel}
\usepackage{graphicx}
%----------------------------------------------------------
\newtheorem{theorem}{Theorem}
\newtheorem{acknowledgement}[theorem]{Acknowledgement}
\newtheorem{algorithm}[theorem]{Algorithm}
\newtheorem{axiom}[theorem]{Axiom}
\newtheorem{case}[theorem]{Case}
\newtheorem{claim}[theorem]{Claim}
\newtheorem{conclusion}[theorem]{Conclusion}
\newtheorem{condition}[theorem]{Condition}
\newtheorem{conjecture}[theorem]{Conjecture}
\newtheorem{corollary}[theorem]{Corollary}
\newtheorem{criterion}[theorem]{Criterion}
\newtheorem{definition}[theorem]{Definition}
\newtheorem{example}[theorem]{Example}
\newtheorem{exercise}[theorem]{Exercise}
\newtheorem{lemma}[theorem]{Lemma}
\newtheorem{notation}[theorem]{Notation}
\newtheorem{problem}[theorem]{Problem}
\newtheorem{proposition}[theorem]{Proposition}
\newtheorem{remark}[theorem]{Remark}
\newtheorem{solution}[theorem]{Solution}
\newtheorem{summary}[theorem]{Summary}
\newenvironment{proof}[1][Proof]{\textbf{#1.} }{\ \rule{0.5em}{0.5em}}
%----------------------------------------------------------
\begin{document}

\title{Dokumentation Computergrafik 1 - Aufgabe 1}
\author{Jan Rabe 766212}
\date{9.11.2010}
\maketitle
\tableofcontents

\part{Aufgabenstellung}

\chapter{Pflichtaufgaben}

Implementieren Sie zun�chst die Rasterisierung von Linie, gef�lltem Kreis, gef�lltem Dreieck und
gef�lltem Polygon in einer einheitlichen Farbe Ihrer Wahl (au�er Schwarz oder dunkles Grau). Implementieren
Sie f�r das Polygon sowohl eine Rasterisierung als ganzes, als auch eine Triangulierung.
Bei der Triangulierung ist es ausreichend, wenn Sie konvexe Polygone zerlegen k�nnen.
Erstellen Sie anschlie�end einen zweiten Rasterisierer, der lediglich die R�nder der Primitiven in wei�er
Farbe zeichnet.
Erweitern Sie nun den ersten Rasterisierer und verwenden Sie die aus der Datei geladenen Farben
zur F�llung der Primitiven. Hierbei muss eine Interpolation der Farben zwischen den Eckpunkten
stattfinden.

\chapter{Ausgef�llte Fl�chen}

Im Rasterzier landen die Koordinaten und Farbwerte als 'Figure'. Dort wird jeweils getestet, ob die Figure ein Rechteck, eine Linie, ein Kreis, ein Dreieck oder ein Polygon ist. Falls es sich um eine Linie handelt, wird diese in das Objekt Line gecasted. Daraufhin f�llt die Linie den FrameBuffer. Der FrameBuffer ist ein zwei-Dimensionaler Color Array, welcher am Ende auf ein Canvaz abgebildet wird. Das hei�t, die Linie zeichnet sich selbst. So auch analog die anderen Fl�chen.\\ \\
Um auf Sonderf�lle einzugehen, habe ich unterschiedliche Test-Dateien mit Koordinaten zu den einzelnen Fl�chen angelegt.

\section{Rechteck}
Die funktionalit�t war bereits gegeben.

\section{Linie}
Anfangs habe ich die Linie mithilfe der Liniengleichung gezeichnet, jedoch haben lediglich zwei F�lle funktioniert. N�mlich von links nach rechts auf Horizontaler Richtung und von oben Links nach unten Rechts in einem 45� Winkel. Daraufhin habe ich etwas rechechiert und bin auf den Bresenham Algorythmus gesto�en. (http://www.cs.unc.edu/~mcmillan/comp136/Lecture6/Lines.html) Dieser geht auch elegant auf alle M�glichen Sonderf�lle ein.

\section{Kreis}
Der Algorythmus zeichnet Zeile f�r Zeile, wobei jeder Punkt �berpr�ft wird, ob er innerhalb des Kreises liegt. Wenn ja wird der FrameBuffer mit dem Farbwert des Kreises gef�llt.

\section{Dreieck}
Anfangs wird der minimale und maximale x-/y-Wert berechnet. Damit hat man das Grenz-Rechteck in welchem sich das Dreieck befindet. Dazu berechnet man Alpha, Beta und Gamma, welche zusammen 1 ergeben. Alpha, Beta und Gamma berechnet man mit Hilfe der Liniengleichung. Nun schaut man nur noch ob, Alpha, Beta und Gamma jeweils im Bereich zwischen 0 und 1 liegen. Wenn ja, liegt der Punkt im Dreieck und wird gezeichnet.

\section{Polygon}
Es gibt die Methode scanFill() f�r den Scan-Fill Algorythmus und fillConvex() f�r den Triangulierungs Algyrthmus.

\subsection{Scan-Fill}
�berpr�ft, ob ein Punkt im Polygon liegt, mit Hilfe des pointInPolygon-Algorythmus. (http://www.ecse.rpi.edu/Homepages/wrf/Research/Short_Notes/pnpoly.html)

\subsection{Triangulation}
Die Idee des Algorythmus von Kong (http://www.sunshine2k.de/stuff/Java/Polygon/Kong/Kong.html) ist es die nach au�enstehenden Ecken (Ears) abzuschneiden. Diese Ecken werden aus der Liste der Polygonknoten entfernt und als Dreieck in die Dreick-Liste hinzugef�gt. Der Kong Algorythmus wird im Konstruktor ausgef�ht und muss daher nur einmal beim erstellen ausgef�hrt werden.

\chapter{Umrandung der Fl�chen}
Um die Umrandungen zu testen, wurde die Klasse OutlineRasterizer erstellt, die von den Figuren die outline()-Methode anstelle der fill()-Methode aufruft.

\section{Rechteck}
Man nehme alle Eckpunkte und verbindet diese durch Linien in vorgegebener Reihenfolge.

\section{Linie}
Bei der Linie kommt nichts hinzu.

\section{Kreis}
Hier habe ich einfach eine weitere Abfrage eingebaut, welche �berpr�ft, ob (-Radius * Math.PI) <= der Liniengleichung liegt. Optimierungsbedarf besteht.

\section{Dreieck}
Analog zum Viereck.

\section{Polygon}
Analog zum Viereck.

\chapter{Vorschl�ge zur Erweiterung}

\section{Suchen Sie nach optimierten Algorithmen und implementieren Sie diese.}
Die Linie funktioniert nun interpoliert in alle Richtungen, dank des Jack Bresenham Line Algorythmus.
\section{Implementieren Sie Antialiasing.}
Aus Zeitmangel, habe ich dies nicht mehr geschafft.
\section{Erweitern Sie die Triangulierung so, dass Sie beliebige Polygone zerlegen k�nnen.}
Mit Hilfe des Kong-Algorythmus, k�nnen nun konkave Polygone rasterisiert werden. Jedoch Polygone, die sich �berschneiden und somit Artefakte bilden, oder Polygone mit L�chern, funktionieren damit jedoch noch nicht. \\

\end{document}
